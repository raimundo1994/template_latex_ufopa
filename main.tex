\documentclass[
	% -- opções da classe memoir --
	12pt,				% tamanho da fonte
	openright,			% capítulos começam em pág ímpar (insere página vazia caso preciso)
	oneside,			% para impressão em frente e verso. Oposto a oneside
	a4paper,			% tamanho do papel.
	% -- opções da classe abntex2 --
	chapter=TITLE,		% títulos de capítulos convertidos em letras maiúsculas
	%section=TITLE,		% títulos de seções convertidos em letras maiúsculas
	%subsection=TITLE,	% títulos de subseções convertidos em letras maiúsculas
	%subsubsection=TITLE,% títulos de subsubseções convertidos em letras maiúsculas
	% -- opções do pacote babel --
	english,			% idioma adicional para hifenização
	french,				% idioma adicional para hifenização
	spanish,			% idioma adicional para hifenização
	brazil				% o último idioma é o principal do documento
	]{abntex2}
	
% ##########################################################
% Pacotes básicos
% ##########################################################

%\usepackage{lmodern}			% Usa a fonte Latin Modern
\usepackage{times}			    % Usa fonte times
\usepackage{mathptmx}			% Usa a fonte Times New Roman
\usepackage[T1]{fontenc}		% Selecao de codigos de fonte.
\usepackage[utf8]{inputenc}		% Codificacao do documento (conversão automática dos acentos)
\usepackage{lastpage}			% Usado pela Ficha catalográfica
\usepackage{indentfirst}		% Indenta o primeiro parágrafo de cada seção.
\usepackage{color}				% Controle das cores
\usepackage{graphicx}			% Inclusão de gráficos
\usepackage{subcaption}				% Inclusão de gráficos lado a lado
\usepackage{microtype} 			% para melhorias de justificação
\usepackage{tabularx,ragged2e}	% Para inserir tabelas
\usepackage{multirow}			% Para mesclar células
\usepackage[dvipsnames,table,xcdraw]{xcolor}		% Permite adicionar cores nas linhas de tabelas
\usepackage{fancyvrb}			% Permite adicionar arquivos de texto
\usepackage[portuguese, ruled, linesnumbered]{algorithm2e} % Uso de algoritmos
\usepackage{amsfonts}			% Permite usar notação de conjuntos
\usepackage{amsmath}			% Permite citar equações
\usepackage{amsthm}				% Permite criar teoremas e experimentos
\usepackage[font={bf, small}, labelsep=endash, labelfont=bf]{caption}	% Faz legenda de figuras ficarem em negrito
\usepackage{cancel}				% Permite fazer expressão tendendo a zero
\usepackage{epstopdf}			% Converte eps para pdf
\usepackage[final]{pdfpages}

\newcolumntype{L}{>{\RaggedRight\arraybackslash}X}
% ---
% ##########################################################
% Pacotes adicionais, usados apenas no âmbito do Modelo Canônico do abnteX2
% ##########################################################

\usepackage{lipsum}				% para geração de dummy text
% ---
% ##########################################################
% Pacotes de citações
% ##########################################################

%\usepackage[brazilian,hyperpageref]{backref}	 % Paginas com as citações na bibl
\usepackage[alf, abnt-emphasize=bf]{abntex2cite}	% Citações padrão ABNT

% ##########################################################
% Customizações para o layout da UFOPA
% ##########################################################

\usepackage{modelo-ufopa/ufopa}
% Muda o título de lista de ilustrações para lista de figuras
\addto\captionsbrazil{%
  \renewcommand{\listfigurename}%
    {Lista de Ilustrações}%
	\renewcommand{\listtablename}%
    {Lista de Tabelas}%
}
% Permite utilizar figuras sem precisar colocar o caminho absoluto
\graphicspath{{imagens/}}
% Define o ambiente de experimentos
\theoremstyle{definition}
\newtheorem{experimento}{Experimento}[section]
\newcommand{\experimentoautorefname}{Experimento}


% ##########################################################
% Informações do TRABALHO
% ##########################################################

\universidade{UNIVERSIDADE FEDERAL DO OESTE PARÁ}
\instituto{INSTITUTO DE ENGENHARIA E GEOCIÊNCIAS}
\faculdade{BACHARELADO EM GEOFÍSICA}
%\curso{CURSO DE BACHARELADO EM SISTEMAS DE INFORMAÇÃO}
\titulo{TÍTULO DA MONOGRAFIA}
\autor{NOME SOBRENOME}
\local{Santarém}
\data{2020}
\orientador{Prof. Dr. Nome Sobrenome}
\coorientador{Prof. Dr. Nome Sobrenome}
\tipotrabalho{Monografia}

% ##########################################################
% o nome da instituição e a área de concentração
% ##########################################################

\preambulo{Monografia apresentada ao Curso de Bacharelado em Geofísica do Instituto de Engenharia e Geociências da Universidade Federal do Oeste do Pará, como requisito parcial para a obtenção do grau de Bacharel em Geofísica.}
\sobrenome{Sobrenome}
\nome{Nome}
\palavraschave{%
1. Geociências.
2. Geologia.
3. Física.
}
\conceito{Conceito:  }
\datadadefesa{Data de Aprovação: }%07 de Dezembro de 2016}
\faculdadedoorientador{Universidade Federal do Oeste do Pará - UFOPA}
\primeiromembrodabanca{Prof. Dr. Nome Sobrenome}
\faculdadedoprimeiromembrodabanca{Universidade Federal do Oeste do Pará - UFOPA}
\segundomembrodabanca{Prof. Dra. Nome Sobrenome}
\faculdadedosegundomembrodabanca{Universidade Federal do Oeste do Pará - UFOPA}
% -------------------------------------------------------------------------
% ##########################################################
% Configurações de aparência do PDF final
% ##########################################################

% alterando o aspecto da cor azul
\definecolor{blue}{RGB}{41,5,195}
% informações do PDF
\makeatletter
\hypersetup{
     	%pagebackref=true,
		pdftitle={\imprimirtitulo},
		pdfauthor={\imprimirautor},
    	pdfsubject={\imprimirpreambulo},
	    pdfcreator={LaTeX with abnTeX2},
		pdfkeywords={\imprimirpalavraschave},
		colorlinks=true,       		% false: boxed links; true: colored links
    	linkcolor=black,          	% color of internal links
    	citecolor=black,        		% color of links to bibliography
    	filecolor=magenta,      		% color of file links
		urlcolor=blue,
		bookmarksdepth=4,
        breaklinks=true
}
\makeatother

% ##########################################################
% Espaçamentos entre linhas e parágrafos
% ##########################################################

% O tamanho do parágrafo é dado por:
\setlength{\parindent}{1.3cm}
% Controle do espaçamento entre um parágrafo e outro:
\setlength{\parskip}{0.2cm}  % tente também \onelineskip
% compila o indice
% ---
\makeindex
% ---

% #########################################################################
% ---------------------------INICIO DO DOCUMENTO---------------------------
% #########################################################################

\begin{document}
% Seleciona o idioma do documento (conforme pacotes do babel)
\selectlanguage{brazil}
% Retira espaço extra obsoleto entre as frases.
\frenchspacing
% ##########################################################
% ELEMENTOS PRÉ-TEXTUAIS
% ##########################################################

% \pretextual

% ##########################################################--
% Capa
% ##########################################################

\imprimircapa
% ---

% ##########################################################
% Folha de rosto
% ##########################################################

\imprimirfolhaderosto
% ---

% ##########################################################
% Inserir a ficha bibliografica
% ##########################################################

\newpage

\begin{fichacatalografica}
	\imprimirfichacatalografica
\end{fichacatalografica}
% ---
% ##########################################################
% Inserir errata ---- ative as linhas abaixo, caso tenha uma errata -----
% ##########################################################
\begin{errata}
Elemento opcional da \citeonline[4.2.1.2]{NBR14724:2011}. Exemplo:

\vspace{\onelineskip}

FERRIGNO, C. R. A. \textbf{Tratamento de neoplasias ósseas apendiculares com
reimplantação de enxerto ósseo autólogo autoclavado associado ao plasma
rico em plaquetas}: estudo crítico na cirurgia de preservação de membro em
cães. 2011. 128 f. Tese (Livre-Docência) - Faculdade de Medicina Veterinária e
Zootecnia, Universidade de São Paulo, São Paulo, 2011.

\begin{table}[htb]
\center
\footnotesize
\begin{tabular}{|p{1.4cm}|p{1cm}|p{3cm}|p{3cm}|}
  \hline
   \textbf{Folha} & \textbf{Linha}  & \textbf{Onde se lê}  & \textbf{Leia-se}  \\
    \hline
    1 & 10 & auto-conclavo & autoconclavo\\
   \hline
\end{tabular}
\end{table}

\end{errata}
% ---

% ##########################################################
% Inserir folha de aprovação
% ##########################################################

% Isto é um exemplo de Folha de aprovação, elemento obrigatório da NBR
% 14724/2011 (seção 4.2.1.3). Você pode utilizar este modelo até a aprovação
% do trabalho. Após isso, substitua todo o conteúdo deste arquivo por uma
% imagem da página assinada pela banca com o comando abaixo:
%
% \includepdf{folhadeaprovacao_final.pdf}
%
\begin{folhadeaprovacao}
\imprimirfolhadeaprovacao
\end{folhadeaprovacao}
% ---

% ##########################################################
% Dedicatória
% ##########################################################

\begin{dedicatoria}
   \vspace*{\fill}
   \begin{flushright}
	   \noindent
	   \textit{Este trabalho é dedicado às crianças adultas que,\\
	   quando pequenas, sonharam em se tornar cientistas.}
	 \end{flushright}
\end{dedicatoria}
% ---

% ##########################################################
% Agradecimentos
% ##########################################################

\begin{agradecimentos}
Os agradecimentos principais são direcionados à Gerald Weber, Miguel Frasson,
Leslie H. Watter, Bruno Parente Lima, Flávio de Vasconcellos Corrêa, Otavio Real
Salvador, Renato Machnievscz\footnote{Os nomes dos integrantes do primeiro
projeto abn\TeX\ foram extraídos de
\url{http://codigolivre.org.br/projects/abntex/}} e todos aqueles que
contribuíram para que a produção de trabalhos acadêmicos conforme
as normas ABNT com \LaTeX\ fosse possível.

Agradecimentos especiais são direcionados ao Centro de Pesquisa em Arquitetura
da Informação\footnote{\url{http://www.cpai.unb.br/}} da Universidade de
Brasília (CPAI), ao grupo de usuários
\emph{latex-br}\footnote{\url{http://groups.google.com/group/latex-br}} e aos
novos voluntários do grupo
\emph{\abnTeX}\footnote{\url{http://groups.google.com/group/abntex2} e
\url{http://www.abntex.net.br/}}~que contribuíram e que ainda
contribuirão para a evolução do \abnTeX.

\end{agradecimentos}
% ---

% ##########################################################
% Epígrafe
% ##########################################################

\begin{epigrafe}
    \vspace*{\fill}
	\begin{flushright}
		\textit{``A verdadeira viagem de descobrimento
        \\não consiste em procurar novas paisagens,
        \\mas em ter novos olhos.''\\
		(Marcel Proust)}
	\end{flushright}
\end{epigrafe}
% ---

% ##########################################################
% RESUMOS
% ##########################################################

% resumo em português
\setlength{\absparsep}{18pt} % ajusta o espaçamento dos parágrafos do resumo
\begin{resumo}
Segundo a \citeonline[3.1-3.2]{NBR6028:2003}, o resumo deve ressaltar o
 objetivo, o método, os resultados e as conclusões do documento. A ordem e a extensão destes itens dependem do tipo de resumo (informativo ou indicativo) e do tratamento que cada item recebe no documento original. O resumo deve ser precedido da referência do documento, com exceção do resumo inserido no próprio documento. (\ldots) As palavras-chave devem figurar logo abaixo do resumo, antecedidas da expressão Palavras-chave:, separadas entre si por ponto e finalizadas também por ponto.

 \textbf{Palavras-chave}: latex. abntex. editoração de texto.
\end{resumo}

% resumo em inglês
\begin{resumo}[Abstract]
 \begin{otherlanguage*}{english}
   This is the english abstract.

   \vspace{\onelineskip}

   \noindent
   \textbf{Keywords}: latex. abntex. text editoration.
 \end{otherlanguage*}
\end{resumo}


% ---

% ##########################################################
% inserir lista de ilustrações
% ##########################################################
\pdfbookmark[0]{\listfigurename}{lof}
\listoffigures*
\cleardoublepage
% ---

% ##########################################################
% inserir lista de quadros
% ##########################################################
\pdfbookmark[0]{\listofquadrosname}{loq}
\listofquadros*
\cleardoublepage
% ---

% ##########################################################
% inserir lista de tabelas
% ##########################################################
\pdfbookmark[0]{\listtablename}{lot}
\listoftables*
\cleardoublepage
% ---

% ##########################################################
% inserir lista de algoritmos
% ##########################################################
\pdfbookmark[0]{\listalgorithmcfname}{loa}
\imprimirlistadealgoritmos
\cleardoublepage
% ---


% ##########################################################
% inserir lista de abreviaturas e siglas
% ##########################################################
\begin{siglas}
	\item[ABNT] Associação Brasileira de Normas Técnicas
    \item[abnTeX] ABsurdas Normas para TeX
\end{siglas}
% ---

% ##########################################################
% inserir lista de símbolos
% ##########################################################
\begin{simbolos}
	\item[$ \Gamma $] Letra grega Gama
  \item[$ \Lambda $] Lambda
  \item[$ \zeta $] Letra grega minúscula zeta
  \item[$ \in $] Pertence
\end{simbolos}
% ---

% ---
% inserir o sumario
% ---
\pdfbookmark[0]{\contentsname}{toc}
\tableofcontents*
\cleardoublepage
% ---

% ##########################################################
% ELEMENTOS TEXTUAIS
% ##########################################################

\mainmatter

% As linhas "\include" abaixo incluem os capítulos no documento.
% Edite os arquivos "chapxx.tex" de acordo com as suas necessidades.
% No presente documento, são incluídos quatro capítulos, mas é possível
% utilizar quantos capítulos forem necessários.

\chapter{Seção primária}

\section{Seção secundária}

\subsection{Seção terciária}

\subsubsection{Seção quarternária}
\chapter{Seção primária}

\section{Seção secundária}

\subsection{Seção terciária}

\subsubsection{Seção quarternária}
\chapter{Seção primária}

\section{Seção secundária}

\subsection{Seção terciária}

\subsubsection{Seção quarternária}
\chapter{Seção primária}

\section{Seção secundária}

\subsection{Seção terciária}

\subsubsection{Seção quarternária}

%\backmatter


% ##########################################################
% ELEMENTOS PÓS-TEXTUAIS
% ##########################################################

\postextual
% ----------------------------------------------------------

% ##########################################################
% Referências bibliográficas
% ##########################################################

\bibliography{bibliografia}
% ---


% ##########################################################
% Apêndices
% ##########################################################

% ---
% Inicia os apêndices
% ---
\begin{apendicesenv}

	% Imprime uma página indicando o início dos apêndices
	\partapendices

% ----------------------------------------------------------
	\chapter{XXXXXXXXXXX}
	\label{cap:apêndices_a}
	
	\chapter{XXXXXXXXXXX}
	\label{cap:apêndices_b}
% ----------------------------------------------------------



\end{apendicesenv}
% ---

% ##########################################################
% Anexos
% ##########################################################
\begin{anexosenv}

	% Imprime uma página indicando o início dos anexos
	\partanexos

	% ---
    \chapter{XXXXXXXXXXX}
    \label{cap:anexo}


\end{anexosenv}

\end{document}

% #########################################################################
% ---------------------------FIM DO DOCUMENTO---------------------------
% #########################################################################
